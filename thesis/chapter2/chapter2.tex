\providecommand{\topdir}{..}
\documentclass[../main.tex]{subfiles}

\ifSubfilesClassLoaded{
    \externaldocument[main-]{../main}
    \setcounterref{chapter}{main-chap:rayleigh_benard}
    \addtocounter{chapter}{-1}
}{}

\begin{document}

\ifSubfilesClassLoaded{
    \frontmatter
    \tableofcontents
    \mainmatter
}{}

\chapter{\rb{} convection} \label{chap:rayleigh_benard}
\setlength{\epigraphwidth}{.45\textwidth}
\epigraphhead[0.1\textheight]{ \epigraph{%
        Good approximations often lead to better ones. }{\emph{Mathematical
    Methods in Science}\\George P\`{o}lya, 1977} }

\section{Dynamical system case study: \rb{} convection}
In the final section of
this review, I will present a case study on the numerical modelling of \rb{}
convection, arguing that it is an ideal intermediate-complexity dynamical
system for further parametrisation development and testing. After reviewing the
necessary preliminary material, I will review the relevant literature to
establish that under-resolved numerical solutions of the system exhibit
systematic errors that may be attributed to the neglect of subgrid-scale
dynamics. An understanding of the nature of these errors will allow future work
to use the \rb{} problem as a parametrisation testbed analogous to L96, and
inform the construction and testing of data-driven schemes.


\subsection{Problem statement}
\rb{} convection is the motion of a fluid confined between two horizontal
isothermal plates, the temperature of the bottom plate being higher than that
of the top plate. The governing equations for the flow follow from the
Navier-Stokes equations of mass, energy and momentum conservation. The reader
is referred to \textcite{chandrasekhar1961} for a detailed derivation; I
summarise the assumptions and approximations involved below.

The density, $\rho$, of the fluid is related to its temperature $T$ by the
linear equation of state
\[
    \rho = \rho_0 [1 - \alpha(T - T_0)],
\]
where $\alpha$ is the (constant) volume coefficient of thermal expansion and
$\rho_0$ and $T_0$ are the base-state density and temperature such that $\rho =
\rho_0$ when $T = T_0$. The key assumption is that density variations are small
($\alpha (T - T_0) \ll 1$), which allows the governing equations to be
simplified under the \emph{Boussinesq approximation}. The Boussinesq
approximation involves first writing the pressure, $p$, of the fluid as
\[
    p = p_0 - \rho_0 gz + p',
\]
where $p_0$ is an arbitrary constant, $g$ is the acceleration due to gravity
and $z$ is the vertical coordinate. $p'$ is the (time-varying) deviation from a
hydrostatically balanced background profile $p_0 - \rho_0 gz$ in which the
upward pressure gradient force per unit volume $\rho_0 g$ cancels the downward
weight force per unit volume $-\rho_0 g$. Since $\alpha (T - T_0) \ll 1$,
density variations are neglected everywhere except in their contribution to the
weight force, leading to a net buoyant (background pressure gradient plus
weight) force per unit mass
\[
    \frac{\rho_0 - \rho}{\rho_0} g = \alpha (T - T_0) g.
\]

With these assumptions in mind, I adopt the governing equations as they are
derived by \textcite{chandrasekhar1961}:
\begin{alignat}{2}
    \label{eqn:dim_momentum}
    \pdiff{\vec{u}}{t} + \vec{u} \cdot \grad \vec{u} &= -\frac{1}{\rho_0} \grad
        p' + \alpha (T - T_0) g \uvec{z} + \nu \nabla^2 \vec{u} &\quad&
        \text{(momentum conservation),} \\
    \label{eqn:dim_energy}
    \pdiff{T}{t} + \vec{u} \cdot \grad T &= \kappa \nabla^2 T && \text{(energy
        conservation), and} \\
    \label{eqn:dim_incompressible}
    \grad \cdot \vec{u} &= 0 && \text{(incompressibility).}
\end{alignat}
$\vec{u}$ is the fluid velocity, $t$ is time, $\uvec{z}$ is the upward unit
vector, $\nu$ is the (constant) kinematic viscosity and $\kappa$ is the thermal
diffusivity (also constant). Notice that the aforementioned buoyancy term
$\alpha (T - T_0) g$ appears on the right-hand side of \cref{eqn:dim_momentum}.

The parametrisation test-bed developed in this work solves the governing
equations in a two-dimensional domain $(x,z) \in [0, d] \times [0, L]$, subject
to no-slip, isothermal boundary conditions on the top and bottom plates,
\begin{alignat}{3}
    \label{eqn:dim_bc_bot}
    \vec{u} &= \vec{0}, &\quad T &= T_0 + \frac{\delta T}{2} &\qquad& \text{at
    } z = 0 \text{ and} \\
    \label{eqn:dim_bc_top}
    \vec{u} &= \vec{0}, &\quad T &= T_0 - \frac{\delta T}{2} &\qquad& \text{at
    } z = d,
\end{alignat}
and periodic boundary conditions in the horizontal,
\begin{alignat}{2}
    \label{eqn:dim_bc_sides}
    \vec{u}(x=0) &= \vec{u}(x=L) &\quad \text{and} \quad T(x=0) &= T(x=L).
\end{alignat}
$\delta T$ is the constant temperature difference between the plates.

\subsection{Nondimensionalisation and scale analysis}
It is helpful to nondimensionalise the governing equations
\crefrange{eqn:dim_momentum}{eqn:dim_bc_sides}; this is not only useful for
numerical work but also gives insight into the different flow regimes that are
possible. A range of nondimensionalisations are used in fluid dynamics
literature; I adopt a common one \parencite[see,
e.g.,][]{grotzbach1983,ouertatani2008,stevens2010} which is suitable for the
turbulent convective regime.

For low-viscosity, turbulent flow, a suitable time scale is the \emph{free-fall
time} $t_0$, which is the time for a fluid element with constant temperature $T
= T_0 - \delta T$ to fall from the top plate to the bottom plate under the
influence of buoyancy ($-g \alpha \delta T$) alone. It is simple to show that
\[
    t_0 \sim \left( \frac{d}{g \alpha \delta T} \right)^{1/2},
\]
ignoring a factor of $\sqrt{2}$. The obvious length and temperature scales are
the plate separation $d$ and temperature difference $\delta T$, respectively.

Making the substitutions $p'/\rho_0 \to \pi$ and $T - T_0 \to \theta$ in
\crefrange{eqn:dim_momentum}{eqn:dim_bc_sides} and expressing all variables in
units of $t_0$, $d$ and $\delta T$ leads to the dimensionless equations
\begin{align}
    \label{eqn:momentum}
    \pdiff{\vec{u}}{t} + \vec{u} \cdot \grad \vec{u}
        &= -\grad \pi + \left( \frac{\prandtl}{\rayleigh}\right)^{1/2}
        \nabla^2 \vec{u} + \theta \uvec{z}, \\
    \label{eqn:energy}
    \pdiff{\theta}{t} + \vec{u} \cdot \grad \theta
        &= (\rayleigh\,\prandtl)^{-1/2} \, \nabla^2 \theta, \quad \text{and} \\
    \label{eqn:incompressible}
    \grad \cdot \vec{u} &= 0,
\end{align}
with boundary conditions
\begin{gather}
\begin{alignat}{3}
    \label{eqn:bc_bot}
    \vec{u} &= \vec{0}, &\quad \theta &= +\frac{1}{2}
    &\qquad& \text{at } z = 0, \\
    \label{eqn:bc_top}
    \vec{u} &= \vec{0}, &\quad \theta &= -\frac{1}{2}
    &\qquad& \text{at } z = 1,
\end{alignat} \\
\begin{alignat}{2}
    \label{eqn:bc_sides}
    \vec{u}(x=0) &= \vec{u}(x=\Gamma)
    &\quad \text{and} \quad \theta(x=0) &= \theta(x=\Gamma).
\end{alignat}
\end{gather}
There are three dimensionless parameters: the aspect ratio of the domain
\[
    \Gamma \equiv \frac{L}{d},
\]
the \emph{Prandtl number}
\[
    \prandtl \equiv \frac{\nu}{\kappa}
\]
which measures the relative importance of viscosity (momentum diffusivity) and
thermal diffusivity, and the \emph{Rayleigh number}
\[
    \rayleigh \equiv \frac{g \alpha d^3 \delta T}{\kappa \nu}.
\]
The Rayleigh number can be interpreted as the ratio of the time scale for
thermal transport by conduction to the time scale for thermal transport by
convection. It determines the importance of diffusion for the evolution of
$\vec{u}$ and $\theta$; inspection of \cref{eqn:momentum,eqn:energy} indicates
that low $\rayleigh$ implies strong diffusion and high $\rayleigh$ weak
diffusion. Detailed theoretical analysis of the governing equations (see, e.g.,
\textcite{chandrasekhar1961} and the seminal work by \textcite{rayleigh1916})
reveals that there exists a critical Rayleigh number (dependent on boundary
conditions but of order $\SI{e3}{}$), below which the equations have a stable
equilibrium with the fluid at rest and a linear conductive temperature profile.
Above the critical value, the equilibrium is unstable and small perturbations
lead to the formation of a regular series of steady, rotating convection cells.
If the Rayleigh number is increased much further (\textcite{le_quere1991} cites
$\rayleigh \approx \SI{2e8}{}$), the solution becomes unsteady and increasingly
turbulent. This work is concerned with the turbulent regime, since Rayleigh
numbers for atmospheric deep moist convection can be as large as $\SI{e22}{}$
\parencite{chilla2012}.

\subsection{Thermal properties}
Two more definitions are necessary before proceeding. First, the Nusselt number
measures the rate of (vertical) heat transport across a horizontal plane at
height $z$, normalised by the purely conductive rate that would exist if the
fluid were at rest \parencite{verzicco1999}. Following \textcite{chilla2012}, I
use the definition (before nondimensionalisation)
\begin{equation}
    \label{eqn:dim_nusselt}
    \nusselt(z,t) \equiv \frac{
        \langle wT \rangle_{A,t}
        - \kappa \partial \langle T \rangle_{A,t} / \partial z
    }{
        \kappa \delta T / d
    }
\end{equation}
where $\langle \cdot \rangle_{A,t}$ denotes averaging over time and the
horizontal plane at height $z$, and $w = \vec{u} \cdot \uvec{z}$ is the
vertical velocity. The rate of heat transport in the numerator has two terms:
advection $\langle wT \rangle_{A,t}$ and conduction $-\kappa \partial \langle T
\rangle_{A,t} / \partial z$. The denominator $\kappa \delta T / d$ is the rate
of heat transport for a linear conductive temperature profile with the fluid at
rest.

Another important quantity is the thickness $\delta_T$ of the \emph{thermal
boundary layer} at each plate where large temperature gradients exist.
\textcite{chilla2012} define $\delta_T$ as follows: if, on average, the fluid
temperature changes with height from $+\delta T/2$ at the lower plate to $0$
(the mean value in the well-mixed interior) over a distance $\delta_T$, then
\[
    \left. \pdiff{\langle T \rangle_{A,t}}{z} \right|_{z=0}
        \approx -\frac{\delta T}{2 \delta_T}.
\]
But if one considers the definition of the Nusselt number
\cref{eqn:dim_nusselt} at $z=0$, the advection term $\langle wT \rangle_{A,t}$
vanishes due to the $\vec{u} = \vec{0}$ boundary condition and
\[
    \nusselt(z=0) = -\frac{d}{\delta T}
        \left. \pdiff{\langle T \rangle_{A,t}}{z} \right|_{z=0}.
\]
Thus,
\begin{equation}
    \label{eqn:thermal_bl}
    \delta_T = \frac{d}{2\,\nusselt(z=0)}.
\end{equation}


\subsection{Resolution dependence of numerical solutions}
% s.sherwood: Might be important to further explain here that you will regard
% this as the ("warts and all") dynamical system you want to emulate at a
% coarser representation.  This avoids the issue of whether this hig-res is a
% perfect solution of the fluid equations or not (it doesn't have to be,
% although for robustness we want it to be relatively insensitive to small
% changes in dx).
The first basic requirement for a parametrisation testbed is a reasonably
accurate high-resolution model to treat as truth. The next requirement is a
truncated coarse model which posesses systematic biases relative to truth that
might reasonably be improved by parametrising the unresolved subgrid-scale
dynamics. In this section, I review relevant literature on numerical solutions
of the \rb{} problem with the aim of establishing the nature of the biases that
might be expected. Practically, the following questions must be answered:
\begin{itemize}
    \item What resolution is necessary for a converged solution?
    \item Which quantities are most sensitive to insufficient resolution?
\end{itemize}


\subsubsection{Theoretical resolution requirements for accurate simulations}%
\label{sec:res_requirements}

% resolution in general, Grotzbach condition.
\textcite{grotzbach1983} is recognised as the first to formulate resolution
requirements for accurate simulations of \rb{} convection
\parencite{chilla2012,scheel2013}. He identified separate constraints on the
mean (i.e., averaged in each spatial direction) grid spacing and the vertical
spacing near the plates; I first discuss the former. \citeauthor{grotzbach1983}
reasoned that a numerical model that neglects subgrid-scale effects must have a
geometric mean grid spacing $h = (\Delta x \Delta y \Delta z)^{1/3}$ such that
\begin{equation}
    \label{eqn:grotzbach}
    h \leq \pi \eta = \pi \left(
        \frac{\nu^3}{\langle \epsilon \rangle}
    \right)^{1/4}
\end{equation}
where $\eta \equiv (\nu^3/\langle \epsilon \rangle)^{1/4}$ is the
\emph{Kolmogorov length}, the universal smallest relevant length scale for
general turbulent flow, and $\langle \epsilon \rangle$ is the spatial and
temporal average of the kinetic energy dissipation rate defined by
\begin{equation}
    \label{eqn:kinetic_dissipation}
    \epsilon(\vec{x}, t) \equiv \frac{\nu}{2} \sum_{ij} \left(
        \pdiff{u_i}{x_j} + \pdiff{u_j}{x_i}
    \right)^2
\end{equation}
\parencite{chilla2012}. The inequality \cref{eqn:grotzbach} between $h$ and
$\eta$ can be understood using the Nyquist-Shannon theorem, which states that a
sampling frequency $f \geq k/\pi$ is needed to unambiguously reconstruct a
signal with maximum wavenumber $k$; substituting $f = 1/h$, $k = 1/\eta$ leads
to the claimed relation.

% resolution near plates, number of points.
\citeauthor{grotzbach1983} recognised that the above reasoning was only valid
for the mean grid spacing; large gradients in temperature and velocity near the
top and bottom plates require finer resolution in those regions. The notion of
nearness can be formalised using the thermal boundary layer thickness
\cref{eqn:thermal_bl}, and one asks how many grid points are necessary in this
layer. \citeauthor{grotzbach1983} did not give a theoretical argument to derive
this number but claimed that 3 points are sufficient for turbulent flows.
\textcite{shishkina2010} presented a theoretical argument based on the
(experimentally and numerically justified) assumption of laminar
\emph{Prandtl-Blasius} flow conditions in the boundary layer and were able to
calculate the minimum number of grid points (e.g., 9 for $\rayleigh =
\SI{2e9}{}$ and $\prandtl = 0.7$). The results agreed with criteria derived in
previous numerical experiments. Importantly, the results of
\citeauthor{shishkina2010} allow \emph{a priori} determination of vertical
resolution requirements, potentially bypassing the time-consuming and expensive
process of iteratively running simulations, checking their convergence and
updating the resolution.

\subsubsection{ Resolution-dependence tests and consequences of
    under-resolution}%
\label{sec:res_tests}

Performing numerical experiments for a 3D fluid layer,
\citeauthor{grotzbach1983} found that RMS velocity and Nusselt number were the
most sensitive quantities to insufficient mean grid spacing, but even they
increased ``only slightly'' above the values obtained from well-resolved
simulations. He concluded that condition \cref{eqn:grotzbach} was overly
restrictive and recommended (for $\prandtl > 0.59$) the simplified, approximate
version
\[ % is this necessary?
    h \lesssim 5.23 \, \prandtl^{-1/4} \rayleigh^{-0.3205}. \] Later work also
supports the finding that the Nusselt number is sensitive to under-resolution.
Even studying only steady-state convective solutions at moderate Rayleigh
number, \textcite{le_quere1991} found that the maximum and minimum Nusselt
numbers were most sensitive to changes in resolution and had the largest
uncertainty among existing benchmark solutions.
% ouertatani: common to use Nu as convergence test
Other studies have used the convergence of the Nusselt number as an indicator
that the spatial resolution is sufficient to produce an accurate solution
\parencite{ouertatani2008}.

\textcite{stevens2010} performed 3D simulations in a finite cylindrical cavity
with the aim of reconciling the apparent disagreement between the Nusselt
numbers in previous numerical studies and experimental observations. They found
that agreement with experiment could be achieved, but only by using a much
higher resolution than the previous studies. They offered the physical
explanation that horizontally under-resolved simulations produce insufficient
thermal diffusion, leading to systematic overestimation of the buoyancy of
convective plumes near the side-walls of the cylinder; this results in Nusselt
numbers that exceed experimentally observed values. This led them to conclude
that the two criteria of \textcite{grotzbach1983}---for mean grid spacing and
for the vertical spacing near the upper and lower plates---are not independent;
the definition $h = (\Delta x \Delta y \Delta z)^{1/3}$ in \cref{eqn:grotzbach}
allows the horizontal spacing to remain relatively coarse near the plates,
provided the vertical spacing is small. Since fine horizontal resolution is
also necessary to accurately capture the dynamics of the thin plumes, they
proposed that \cref{eqn:grotzbach} be applied with $h = \max(\Delta x, \Delta
y, \Delta z)$ instead.

% kooij: Nu may not be best indicator
Some more recent work, however, casts doubt on the notion that the Nusselt
number is sensitive to under-resolution and that its convergence is a good
indicator that the flow is well-resolved. In assessing the performance of
several published computational fluid dynamics codes on the \rb{} problem in a
cylindrical cavity, \textcite{kooij2018} identified one higher-order code that
reproduced the theoretically predicted scaling of $\nusselt$ as a function of
$\rayleigh$ even when the flow was deliberately under-resolved. On the other
hand, the presence of numerical artefacts in the instantaneous temperature
field near the bottom plate was a clear indicator of insufficient resolution.


% scheel: dissipation rates sensitive
\textcite{scheel2013} performed similar high-resolution simulations for a
cylindrical cavity and also found that the Nusselt number, among other global
transport properties, were ``fairly insensitive to insufficient resolution, as
long as the mean Kolmogorov length [was] resolved'' (i.e., \cref{eqn:grotzbach}
was satisfied). However, they found that the horizontally averaged or local
kinetic energy dissipation rate \cref{eqn:kinetic_dissipation} and the
corresponding thermal dissipation rate
\begin{equation}
    \label{eqn:thermal_dissipation}
    \epsilon_T(\vec{x}, t) \equiv \kappa \sum_i \left(\pdiff{T}{x_i}\right)^2
\end{equation}
were much more sensitive, with their convergence requiring even stricter
conditions than \cref{eqn:grotzbach}.

In summary, conditions on the mean grid spacing and number of grid points in
the thermal boundary layer exist to guide high-resolution simulations. It is
known that the Nusselt number and kinetic and thermal dissipatiom rates are
sensitive to under-resolution, so the statistics of these quantities could
serve as metrics for evaluating parametrisation performance.


\clearpage
\section{Summary}

\ifSubfilesClassLoaded{%
    \emergencystretch=5em
    \printbibliography{}
}{}

\end{document}
