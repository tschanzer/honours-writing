\providecommand{\topdir}{..}
\documentclass[../main.tex]{subfiles}

\ifSubfilesClassLoaded{
    \externaldocument[main-]{../main}
    \externaldocument[fm-]{../00_front_matter/front_matter}
    \externaldocument[intro-]{../01_introduction/introduction}
    \externaldocument[l96-]{../02_lorenz96/lorenz96}
    \externaldocument[tend-]{../04_tendencies/tendencies}
    \externaldocument[eval-]{../05_evaluation/evaluation}
    \externaldocument[conc-]{../06_conclusion/conclusion}

    \setcounterref{chapter}{main-chap:rayleigh_benard}
    \addtocounter{chapter}{-1}
}{}

\begin{document}

\ifSubfilesClassLoaded{
    \frontmatter
    \tableofcontents
    \mainmatter
}{}

\dropchapter{4.5cm}
\chapter{\rb{} convection} \label{chap:rayleigh_benard}
\setlength{\epigraphwidth}{0.9\linewidth}
\epigraphhead[0.13\textheight]{
    \epigraph{
        \begin{minipage}{0.52\linewidth}
            Talia flammato secum dea corde volutans \\
            nimborum in patriam, loca feta furentibus austris, \\
            Aeoliam venit. Hic vasto rex Aeolus antro \\
            luctantes ventos tempestatesque sonoras \\
            imperio premit ac vinclis et carcere frenat. \\
            Illi indignantes magno cum murmure montis \\
            circum claustra fremunt; celsa sedet Aeolus arce \\
            sceptra tenens, mollitque animos et temperat iras. \\
            Ni faciat, maria ac terras caelumque profundum \\
            quippe ferant rapidi secum verrantque per auras. \\
            Sed pater omnipotens speluncis abdidit atris, \\
            hoc metuens, molemque et montis insuper altos \\
            imposuit, regemque dedit, qui foedere certo \\
            et premere et laxas sciret dare iussus habenas.
        \end{minipage}
        \hspace{\fill}
        \begin{minipage}{0.45\linewidth}
            Thus inwardly brooding with heart inflamed, the goddess came
            to Aeolia, motherland of storm clouds, tracts teeming with
            furious blasts. Here in his vast cavern, Aeolus, their king,
            keeps under his sway and with prison bonds curbs the
            struggling winds and the roaring gales. They, to the
            mountain's mighty moans, chafe blustering around the
            barriers. In his lofty citadel sits Aeolus, sceptre in hand,
            taming their passions and soothing their rage; did he not
            so, they would surely bear off with them in wild flight seas
            and lands and the vault of heaven, sweeping them through
            space. But, fearful of this, the father omnipotent hid them
            in gloomy caverns, and over them piled high mountain masses
            and gave them a king who, under fixed covenant, should be
            skilled to tighten and loosen the reins at command.
        \end{minipage}
        \vspace{.5\baselineskip}
    }{
        Virgil, \emph{Aeneid I}, ll. 50--63 \\
        trans. H. R. Fairclough
    }
}
\undodrop

\todo{introductory paragraph - explain where this is going}

\section{Governing equations}
\rb{} convection is the overturning motion of a fluid confined between two
horizontal isothermal plates, the bottom plate being warmer than the top plate.
The fluid is traditionally assumed to be incompressible, with density $\rho$
related to temperature $T$ by the linear equation of state
\[
    \rho = \rho_0 [1 - \alpha(T - T_0)],
\]
$\alpha$ being the volume coefficient of thermal expansion and $\rho_0$ and
$T_0$ a reference density and temperature. Provided that the density variations
are small ($\alpha (T - T_0) \ll 1$), one may employ the \emph{Boussinesq
approximation}, neglecting density variations everywhere except in their
contribution to the weight force. This leads to the Boussinesq equations
\begin{align}
    \label{eqn:dim_momentum}
    \pdiff{\vec{u}}{t} + \vec{u} \cdot \grad \vec{u}
        &= -\frac{1}{\rho_0} \grad p' + \alpha (T - T_0) g \uvec{z}
            + \nu \nabla^2 \vec{u}, \\
    \label{eqn:dim_energy}
    \pdiff{T}{t} + \vec{u} \cdot \grad T &= \kappa \nabla^2 T,
        \quad \text{and} \\
    \label{eqn:dim_incompressible}
    \grad \cdot \vec{u} &= 0,
\end{align}
which represent the momentum balance, the conservation of energy and the
assumption of incompressiblity, respectively. The prognostic variables are the
fluid velocity $\vec{u}$ and the temperature $T$ (the pressure perturbation
$p'$ is implicitly determined by \cref{eqn:dim_incompressible}). The parameters
of the model are the kinematic viscosity $\nu$ and thermal diffusivity
$\kappa$. $\uvec{z}$ is the upward unit vector. The Boussinesq approximation
has led to the appearance of a buoyant force (per unit mass)
\[
    \alpha (T - T_0) g = \frac{\rho_0 - \rho}{\rho_0} g
\]
in the momentum equation, in agreement with Archimedes' principle. The reader
is referred to \textcite{chandrasekhar1961} for a detailed derivation of the
Boussinesq equations; I have merely summarised the main assumptions and
approximations involved.

In this work, I consider solutions of
\crefrange{eqn:dim_momentum}{eqn:dim_incompressible} in a two-dimensional
domain $[0, d] \times [0, L]$ with Cartesian coordinates $x$ and $z$, subject
to no-slip, isothermal boundary conditions
\begin{alignat}{3}
    \label{eqn:dim_bc_bot}
    \vec{u} &= \vec{0}, &\quad T &= T_0 + \frac{\delta T}{2} &\quad& \text{at
    } z = 0 \text{ and} \\
    \label{eqn:dim_bc_top}
    \vec{u} &= \vec{0}, &\quad T &= T_0 - \frac{\delta T}{2} &\quad& \text{at
    } z = d
\end{alignat}
on the top and bottom plates, and periodic boundary conditions
\begin{alignat}{2}
    \label{eqn:dim_bc_sides}
    \vec{u}(x=0) &= \vec{u}(x=L) &\quad \text{and} \quad T(x=0) &= T(x=L)
\end{alignat}
in the horizontal. $\delta T$ is the constant temperature difference between
the plates.

\section{Nondimensionalisation and scale analysis}
It is helpful to nondimensionalise the governing equations
\crefrange{eqn:dim_momentum}{eqn:dim_bc_sides}; this is not only useful for
numerical work but also gives insight into the different flow regimes that are
possible. A range of nondimensionalisations are used in fluid dynamics
literature; I adopt a common one \parencite[see,
e.g.,][]{grotzbach1983,ouertatani2008,stevens2010} which is suitable for the
turbulent convective regime.

The first step is to choose representative time, length and temperature scales.
For low-viscosity, turbulent flow, a suitable time scale is the \emph{free-fall
time} $t_0$, which is the time for a fluid element with constant temperature $T
= T_0 - \delta T$ to fall from the top plate to the bottom plate under the
influence of buoyancy ($-g \alpha \delta T$) alone. It is simple to show that
\[
    t_0 \sim \left( \frac{d}{g \alpha \delta T} \right)^{1/2},
\]
ignoring a factor of $\sqrt{2}$. The obvious length and temperature scales are
the plate separation $d$ and temperature difference $\delta T$, respectively.

Making the substitutions $p'/\rho_0 \to \pi$ and $T - T_0 \to \theta$ in
\crefrange{eqn:dim_momentum}{eqn:dim_bc_sides} and expressing all variables in
units of $t_0$, $d$ and $\delta T$ leads to the dimensionless equations
\begin{align}
    \label{eqn:momentum}
    \pdiff{\vec{u}}{t} + \vec{u} \cdot \grad \vec{u}
        &= -\grad \pi + \left( \frac{\prandtl}{\rayleigh}\right)^{1/2}
        \nabla^2 \vec{u} + \theta \uvec{z}, \\
    \label{eqn:energy}
    \pdiff{\theta}{t} + \vec{u} \cdot \grad \theta
        &= (\rayleigh\,\prandtl)^{-1/2} \, \nabla^2 \theta, \quad \text{and} \\
    \label{eqn:incompressible}
    \grad \cdot \vec{u} &= 0,
\end{align}
which are solved in the domain $[0, \Gamma] \times [0, 1]$ with boundary
conditions
\begin{gather}
\begin{alignat}{3}
    \label{eqn:bc_bot}
    \vec{u} &= \vec{0}, &\quad \theta &= +\frac{1}{2}
    &\qquad& \text{at } z = 0, \\
    \label{eqn:bc_top}
    \vec{u} &= \vec{0}, &\quad \theta &= -\frac{1}{2}
    &\qquad& \text{at } z = 1,
\end{alignat} \\
\begin{alignat}{2}
    \label{eqn:bc_sides}
    \vec{u}(x=0) &= \vec{u}(x=\Gamma)
    &\quad \text{and} \quad \theta(x=0) &= \theta(x=\Gamma).
\end{alignat}
\end{gather}
There are three dimensionless parameters: the aspect ratio of the domain
\[
    \Gamma \equiv \frac{L}{d},
\]
the \emph{Prandtl number}
\[
    \prandtl \equiv \frac{\nu}{\kappa}
\]
which measures the relative importance of viscosity (momentum diffusivity) and
thermal diffusivity, and the \emph{Rayleigh number}
\[
    \rayleigh \equiv \frac{g \alpha d^3 \delta T}{\kappa \nu}.
\]
The Rayleigh number can be interpreted as the ratio of the time scale for
thermal transport by conduction to the time scale for thermal transport by
convection. It determines the importance of diffusion for the evolution of
$\vec{u}$ and $\theta$; inspection of \cref{eqn:momentum,eqn:energy} indicates
that low $\rayleigh$ implies strong diffusion and high $\rayleigh$ weak
diffusion. Stability analysis (see, e.g., \textcite{chandrasekhar1961} and the
seminal work by \textcite{rayleigh1916}) reveals that there exists a critical
Rayleigh number (dependent on boundary conditions but of order $\SI{e3}{}$),
below which the equations have a stable equilibrium with the fluid at rest and
a linear conductive temperature profile. Above the critical Rayleigh number,
the equilibrium is unstable and small perturbations lead to the formation of a
regular series of steady convection cells. If the Rayleigh number is increased
much further (\textcite{le_quere1991} cites $\rayleigh \approx \SI{2e8}{}$),
the solution becomes unsteady and increasingly turbulent. This work is
concerned with the turbulent regime, since Rayleigh numbers for atmospheric
deep moist convection can be as large as $\SI{e22}{}$ \parencite{chilla2012}.


\section{Large-eddy simulation} \label{sec:les}
In \cref{eval-chap:evaluation} I will compare the output of a parametrised
low-resolution model of \crefrange{eqn:dim_momentum}{eqn:bc_sides} to the
output of the unparametrised base model (i.e., the control). However, it was
found in the early stages of this work that unparametrised low-resolution
models were prone to numerical instability, preventing the obtainment of a
suitable control solution. This is to be expected; at high Rayleigh numbers,
when dissipation is weak, the solutions develop large gradients near
small-scale features that cannot be properly resolved by coarse models, causing
them to crash. To be more precise, the energy spectra of turbulent flows
exhibit an \emph{energy cascade} whereby energy is transferred from
larger-scale motions to smaller-scale motions; only at sufficiently small
scales is energy removed from the system by the dissipative terms in the
equations \parencite{pope2000}. In coarse models where these smallest scales
are not resolved, excess energy accumulates at the highest resolved
wavenumbers.

I choose to stabilise the numerical model by artificially introducing
additional dissipative terms that act on the smallest scales, modelling the
energy transfer to unresolved motions in what is effectively a simple
parametrisation. This is known as \emph{large-eddy simulation} or LES. There
are many options when it comes to choosing the additional terms, but for the
sake of simplicity I adopt the Smagorinsky model, the oldest and simplest. As
it happens, the Smagorinsky model was originally proposed in the context of
atmospheric modelling, by \textcite{smagorinsky1963}. The following description
of the model follows \textcite{pope2000}.

It is easiest to first rewrite \cref{eqn:momentum} and \cref{eqn:energy} in
index notation (with implied summation over repeated indices). The modified
equations read
\begin{align}
    \label{eqn:momentum_idx}
    \pdiff{u_i}{t} + u_j \partial_j u_i
        &= -\partial_i \pi + \left(\frac{\prandtl}{\rayleigh}\right)^{1/2}
            \partial_j \partial_j u_i + \theta \delta_{iz}
            + {\color{red} \partial_j \tau_{ij}} \quad \text{and} \\
    \label{eqn:energy_idx}
    \pdiff{\theta}{t} + u_j \partial_j \theta
        &= (\rayleigh\,\prandtl)^{-1/2} \, \partial_j \partial_j \theta
            + {\color{red} \partial_j F_j},
\end{align}
where the new terms are highlighted in red and $\delta_{ij}$ is the Kronecker
delta symbol. $\tau_{ij}$ is known as the \emph{residual stress tensor} and
$F_j$ is known as the \emph{residual heat flux}. The residual stress tensor is
given by
\[
    \tau_{ij} = 2 \nu_\mathrm{sgs} S_{ij}^*,
\]
where
\[
    \nu_\mathrm{sgs} = (C_S \Delta)^2 \sqrt{2 S_{ij} S_{ij}}
\]
is the so-called \emph{eddy viscosity} and
\[
    S_{ij} = \frac{1}{2} \left(
        \frac{\partial u_i}{\partial x_j} + \frac{\partial u_j}{\partial x_i}
    \right)
\]
is the \emph{strain rate tensor}. $S_{ij}^*$ is the deviatoric part of the
strain rate tensor given by
\[
    S_{ij}^* = S_{ij} - \frac{1}{3} S_{kk} \delta_{ij};
\]
in fact, for incompressible flows $\frac{1}{3} S_{kk} = \partial_i u_i = 0$ so
$S_{ij}^* = S_{ij}$. $\Delta \equiv 2 (\Delta x \Delta z)^{1/2}$ is
a filter width depending on the \emph{local} grid spacings $\Delta x$ and
$\Delta z$, and $C_S = 0.17$ is the Smagorinsky constant.

It is known that the Smagorinsky model introduces too much artificial
dissipation near walls in bounded flows such as \rb{} convection
\parencite{pope2000}. This is remedied by redefining the eddy viscosity to be
\[
    \nu_\mathrm{sgs}
        = (C_S \Delta)^2 (1 - e^{-z^+/A^+})^2 \sqrt{2 S_{ij} S_{ij}},
\]
where the \emph{van Driest damping function} $1 - e^{-z^+/A^+}$ attenuates
the viscosity near the walls. The damping function depends on
\[
    z^+ \equiv \frac{\min(z,1-z)}{\delta_\nu},
\]
the distance to the nearest wall normalised by the viscous length scale
$\delta_\nu$. It is conventional to choose $A^+ = 26$.
\todo{explain measurement of delta-nu}

The residual heat flux is similarly modelled as
\[
    F_j = \kappa_\mathrm{sgs} \partial_j \theta,
\]
where it is assumed that the \emph{eddy diffusivity}
$\kappa_\mathrm{sgs}$ is equal to the eddy viscosity $\nu_\mathrm{sgs}$
\parencite[as in][, for example]{vreugdenhil2018}.


\section{Model configuration}
The nondimensionalised Boussinesq equations
\crefrange{eqn:momentum}{eqn:bc_sides} are solved numerically using Dedalus
(\texttt{v3}), a spectral code built in Python \parencite{burns2020}. Spectral
methods represent the solution of a partial differential equation as a linear
combination of linearly independent basis functions, solving for the
coefficients of the basis function expansion rather than the values of the
solution in real space. The resolution of the model is thus set by choosing the
number of basis functions used. This work uses a Fourier (sine/cosine) basis in
the horizontal direction and a basis of Chebyshev polynomials (of the first
kind) in the vertical direction. The Fourier basis has the special property
that all linear combinations respect the periodic boundary conditions.
Timestepping is performed using a second-order Runge-Kutta scheme. The complete
model source code is publicly available \todo{data availability}.

The Rayleigh number was set to $10^9$, a value at which the flow was found to
be unsteady and feature transient eddies. The aspect ratio of the domain was
$\Gamma = 8$ and the Prandtl number was set to $1$ for simplicity.

\todo{link to animations}


\section{Choice of resolution: literature} \label{sec:resolution}
The next important step is to choose the resolution of the fine and coarse
models. The fine model will serve as ``truth'', providing the training dataset
to which the parametrisation model will be fitted and the test dataset against
which the parametrised coarse model will be evaluated. Its resolution should
therefore be high enough that the (yet-to-be-)chosen evaluation metrics are
relatively insensitive to small changes in resolution (i.e., there should be
near-convergence of long-term statistics). Note, however, that neither a
perfect solution of the fluid equations nor perfect agreement with real-world
experimental measurements are necessary; the fine model merely serves as a
reference dynamical system whose behaviour I aim to reproduce by parametrising
the coarse model. In order to create a worthwhile parametrisation problem, the
coarse model's resolution should therefore be low enough that it exhibits
statistically significant biases relative to the fine model, as measured by the
chosen evaluation metrics. However, as explained in \cref{sec:les}, the
coarse model must still be numerically stable so that it can provide an
unparametrised control solution.

With the above constraints in mind, \cref{sec:res_requirements} will review the
guidelines that have been established in the literature for producing
well-resolved simulations of \rb{} convection. \cref{sec:metrics} will
then review the metrics that are known to be sensitive to under-resolution;
these may be used to experimentally determine appropriate resolutions for the
fine and coarse models, and later (in \cref{eval-chap:evaluation}) to evaluate
parametrisation performance.


\subsection{Resolution guidelines}
\label{sec:res_requirements}

\textcite{grotzbach1983} is recognised as the first to formulate resolution
requirements for accurate simulations of \rb{} convection
\parencite{chilla2012,scheel2013}. He identified separate constraints on the
mean (i.e., averaged in each spatial direction) grid spacing and the vertical
spacing near the plates; I first discuss the former. \citeauthor{grotzbach1983}
reasoned that a numerical model that neglects subgrid-scale effects must have a
geometric mean grid spacing $h = (\Delta x \Delta y \Delta z)^{1/3}$ such that
\begin{equation}
    \label{eqn:grotzbach}
    h \leq \pi \eta = \pi \left(
        \frac{\nu^3}{\langle \epsilon \rangle}
    \right)^{1/4}
\end{equation}
where $\eta \equiv (\nu^3/\langle \epsilon \rangle)^{1/4}$ is the
\emph{Kolmogorov length}, the universal smallest relevant length scale for
general turbulent flow, and $\langle \epsilon \rangle$ is the spatial and
temporal average of the kinetic energy dissipation rate defined by
\begin{equation}
    \label{eqn:kinetic_dissipation}
    \epsilon(\vec{x}, t) \equiv \frac{\nu}{2} \sum_{ij} \left(
        \pdiff{u_i}{x_j} + \pdiff{u_j}{x_i}
    \right)^2
\end{equation}
\parencite{chilla2012}. The inequality \cref{eqn:grotzbach} between $h$ and
$\eta$ can be understood using the Nyquist-Shannon theorem, which states that a
sampling frequency $f \geq k/\pi$ is needed to unambiguously reconstruct a
signal with maximum wavenumber $k$; substituting $f = 1/h$, $k = 1/\eta$ leads
to the claimed relation.

\citeauthor{grotzbach1983} recognised that the above reasoning was only valid
for the mean grid spacing; large gradients in temperature and velocity near the
top and bottom plates require finer resolution in those regions. The notion of
nearness can be formalised by considering the thickness of the \emph{thermal
boundary layer}, the region at each plate where large temperature gradients
exist. Consider a Taylor series expansion of the horizontally- and
time-averaged dimensionless temperature profile about $z=0$; recalling the
boundary condition \cref{eqn:bc_bot}, this reads
\[
    \langle \theta \rangle_{A,t} \approx \frac{1}{2}
        + \left. \pdiff{\langle \theta \rangle_{A,t}}{z} \right|_{z=0} z
        + O(z^2).
\]
The thermal boundary layer thickness $\delta_T$ is defined \parencite[see,
e.g.,][]{chilla2012} as the height at which, to first order, the temperature
reaches the mean value in the well-mixed interior of the domain (i.e., zero for
this problem). It follows that
\begin{equation}
    \label{eqn:thermal_bl}
    \delta_T = -\frac{1}{2} \left(
        \left. \pdiff{\langle \theta \rangle_{A,t}}{z} \right|_{z=0}
    \right)^{-1}.
\end{equation}

One then asks how many grid points must be within the thermal boundary layer.
\citeauthor{grotzbach1983} claimed that 3 points are sufficient for turbulent
flows but did not give a theoretical argument to derive this number.
\textcite{shishkina2010} presented a theoretical argument based on the
(experimentally and numerically justified) assumption of laminar
\emph{Prandtl-Blasius} flow conditions in the boundary layer and were able to
calculate the minimum number of grid points (e.g., 9 for $\rayleigh =
\SI{2e9}{}$ and $\prandtl = 0.7$). The results agreed with criteria derived in
previous numerical experiments. Importantly, the results of
\citeauthor{shishkina2010} allow \emph{a priori} determination of vertical
resolution requirements, potentially bypassing the time-consuming and expensive
process of iteratively running simulations, checking their convergence and
updating the resolution.


\subsection{Metrics sensitive to under-resolution} \label{sec:metrics}
Performing numerical experiments for a 3D fluid layer,
\citeauthor{grotzbach1983} found that the RMS velocity and the \emph{Nusselt
number} were the most sensitive quantities to insufficient mean grid spacing,
but even they increased ``only slightly'' above the values obtained from
well-resolved simulations. The dimensionless Nusselt number measures the
instantaneous rate of (vertical) heat transport across a horizontal plane at
height $z$, normalised by the purely conductive rate that would exist if the
fluid were at rest \parencite{verzicco1999}. Following \textcite{kooij2018}, I
define
\begin{equation}
    \label{eqn:nusselt}
    \nusselt(z,t) \equiv \sqrt{\rayleigh\,\prandtl} \langle w \theta \rangle_A
        - \left\langle \pdiff{\theta}{z} \right\rangle _A
\end{equation}
where $\langle \cdot \rangle_A$ denotes averaging over the horizontal plane at
height $z$, and $w = \vec{u} \cdot \uvec{z}$ is the vertical velocity. Notice
that the heat transport has contribitions from both advection (the first term)
and conduction (the second term). Authors frequently consider the
vertically-averaged value
\[
    \langle \nusselt(z,t) \rangle_z
        = \sqrt{\rayleigh\,\prandtl} \langle w \theta \rangle_{A,z} + 1
\]
and/or the time-averaged value, which is is independent of height due to
energy conservation,
\[
    \langle \nusselt(z,t) \rangle_t = \langle \nusselt(z,t) \rangle_{z,t}
        = \sqrt{\rayleigh\,\prandtl} \langle w \theta \rangle_{A,z,t} + 1.
\]

Later work also supports the finding that the Nusselt number is sensitive to
under-resolution. Even studying only steady-state convective solutions at
moderate Rayleigh number, \textcite{le_quere1991} found that the maximum and
minimum Nusselt numbers were most sensitive to changes in resolution and had
the largest uncertainty among existing benchmark solutions.
Other studies have used the convergence of the Nusselt number as an indicator
that the spatial resolution is sufficient to produce an accurate solution
\parencite{ouertatani2008}.

\textcite{stevens2010} performed 3D simulations in a finite cylindrical cavity
with the aim of reconciling the apparent disagreement between the Nusselt
numbers in previous numerical studies and experimental observations. They found
that agreement with experiment could be achieved, but only by using a much
higher resolution than the previous studies. They offered the physical
explanation that horizontally under-resolved simulations produce insufficient
thermal diffusion, leading to systematic overestimation of the buoyancy of
convective plumes near the side-walls of the cylinder; this results in Nusselt
numbers that exceed experimentally observed values. This led them to conclude
that the two criteria of \textcite{grotzbach1983}---for mean grid spacing and
for the vertical spacing near the upper and lower plates---are not independent;
the definition $h = (\Delta x \Delta y \Delta z)^{1/3}$ in \cref{eqn:grotzbach}
allows the horizontal spacing to remain relatively coarse near the plates,
provided the vertical spacing is small. Since fine horizontal resolution is
also necessary to accurately capture the dynamics of the thin plumes, they
proposed that \cref{eqn:grotzbach} be applied with $h = \max(\Delta x, \Delta
y, \Delta z)$ instead.

Some more recent work, however, casts doubt on the notion that the Nusselt
number is sensitive to under-resolution and that its convergence is a good
indicator that the flow is well-resolved. In assessing the performance of
several published computational fluid dynamics codes on the \rb{} problem in a
cylindrical cavity, \textcite{kooij2018} identified one higher-order code that
reproduced the theoretically predicted scaling of $\nusselt$ as a function of
$\rayleigh$ even when the flow was deliberately under-resolved. On the other
hand, the presence of numerical artefacts in the instantaneous temperature
field near the bottom plate was a clear indicator of insufficient resolution.

\textcite{scheel2013} performed similar high-resolution simulations for a
cylindrical cavity and also found that the Nusselt number, among other global
transport properties, were ``fairly insensitive to insufficient resolution, as
long as the mean Kolmogorov length [was] resolved'' (i.e., \cref{eqn:grotzbach}
was satisfied). However, they found that the horizontally averaged or local
kinetic energy dissipation rate \cref{eqn:kinetic_dissipation} and the
corresponding thermal dissipation rate
\begin{equation}
    \label{eqn:thermal_dissipation}
    \epsilon_T(\vec{x}, t) \equiv \kappa \sum_i \left(\pdiff{T}{x_i}\right)^2
\end{equation}
were much more sensitive, with their convergence requiring even stricter
conditions than \cref{eqn:grotzbach}.


\section{Choice of resolution: experiment}
\cref{sec:resolution} showed that simulations must satisfy certain grid spacing
conditions in order to be accurate, and presented several metrics that are
sensitive to under-resolution. I now perform numerical experiments and
use these conditions and metrics to determine appropriate resolutions for
the fine and coarse models in the parametrisation study that will follow.

\todo{describe experiments and present results}


\ifSubfilesClassLoaded{%
    \emergencystretch=5em
    \printbibliography{}
}{}

\end{document}
